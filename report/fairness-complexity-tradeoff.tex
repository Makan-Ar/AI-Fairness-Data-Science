\documentclass[sigconf]{acmart}

\usepackage{booktabs} % For formal tables
\usepackage{natbib}

% Copyright
%\setcopyright{none}
%\setcopyright{acmcopyright}
%\setcopyright{acmlicensed}
\setcopyright{rightsretained}
%\setcopyright{usgov}
%\setcopyright{usgovmixed}
%\setcopyright{cagov}
%\setcopyright{cagovmixed}


\begin{document}
\title{Fairness and Complexity Trade-off in Machine Learning}
%\titlenote{Produces the permission block, and
%  copyright information}
%\subtitle{}
%\subtitlenote{The full version of the author's guide is available as
%\texttt{acmart.pdf} document}

\author{Anonymized Submission}
%\author{Makan Arastuie}
%\affiliation{%
  %\institution{The University of Toledo}
  %\city{Toledo}\textbf{•}
  %\state{Ohio}
  %\postcode{43607}
%}
%\email{makan.arastuie@rockets.utoledo.edu}


\begin{abstract}
As more machine learning applications are being trusted with making decisions that directly or indirectly affect human lives, the potential lack of fairness or discriminatory decisions have become a growing concern within the machine learning community. Recently, researchers have been designing fair models which have been relatively successful in achieving a specific type of fairness in a specific setting, however the issue is that there is no unique or relatively straight forward way of implementing a fair machine learning algorithm. There is no guarantee that a machine learning model that affects individuals is going to be deployed by an expert computer scientist who is capable of an in depth analysis on the fairness of the model. Thus, in this study we explore how well-known and widely used machine learning supervised classification model perform with respect to 4 different definitions of fairness. More importantly, we empirically analyze how different measures of fairness vary as the complexity of these models increases. We observed a weak trade-off between fairness and complexity, however there were some exceptions to this trend. In addition, we demonstrated how being fair with respect to two of the most widely used definitions of fairness, does not guarantee fairness with respect to some other definitions. We believe that uncovering general trends (e.g. this trade-off) will help developers implement more fair algorithms without too much expertise in this field.
\end{abstract}


\keywords{fairness, complexity, machine learning, fairness vs. complexity, trade-off, demographic parity, equality of opportunity}

\maketitle

\section{Introduction} \label{sec:intro}
Machine learning is no longer an esoteric subject. Over the past decade, due to its ease of implementation and reliable accuracy, it has been adopted in a verity of diverse fields such as finance  \cite{huang2007credit, tsai2008using, galindo2000credit}, crime prediction \cite{brennan2009evaluating} and so on. Decision making in these areas has legal, moral and ethical implications, all of which should be considered while aiming for increasing prediction accuracy. The study of fairness considers the \textit{unjust} systematical discrimination against particular groups of people \cite{angwin2016machine, chouldechova2017fair, barocas2016big, berk2017fairness}. In this context, a non-discriminatory AI is not an AI that does not discriminate, it is an AI that does not cause unjust or unfair discrimination.

Interest in non-discriminatory AI has been increasing not only due to moral reasons, but also due to numerous anti-discriminatory laws in many countries. Discrimination is the prejudicial treatment of an individual based on membership in a legally protected group such as a race or gender \cite{calmon2017optimized}. These laws typically evaluate the fairness of a decision making process by means of two distinct notions: \textbf{disparate treatment} and \textbf{disparate impact} \cite{zafar2017fairness}. Disparate treatment occurs when protected attributes are used explicitly in making decisions, also known as direct discrimination. More pervasive nowadays is disparate impact, in which protected attributes are not used but reliance on variables correlated with them leads to significantly different outcomes for different groups. This can also be labled as indirect discrimination. Indirect discrimination may be intentional, as in the historical practice of \textit{redlining} in the U.S. in which home mortgages were denied in zip codes populated primarily by minorities. However, the doctrine of disparate impact applies regardless of actual intent \cite{calmon2017optimized}.

There is no universal definition of fairness. The purpose of fairness and what it means to be fair differs for every setting and this is what makes this problem extremely hard. As it is explained in Section \ref{sec:background} there are different measures and definitions of fairness and it is possible for an AI to be completely fair with respect to one measure and "unfair" to another. Thus, it is crucial to choose the right definition of fairness that embraces the needs of the project.

Furthermore, there exists an inevitable trade-off between fairness and accuracy in machine learning and almost all studies done in this area are attempts to achieve a reasonable balance in this trade-off. On the other hand, there is also a classical and long studied trade-off between accuracy and complexity in machine learning as well. The trade-off between the two is not always clear. Depending on the dataset it is possible for a simple machine learning algorithm to outperform a more complex one as it was shown by \citet{holte1993very}. However, most often that not, more complex algorithms tend to achieve a higher accuracy \cite{zemel2013learning}.

In this study, we aim to combine the two aforementioned trade-offs. We empirically analyze the trade-off between fairness and complexity. In other words, we are interested to observe how more complex algorithms perform with respect to different fairness measures. One challenging aspect of this problem is ranking the complexity of various machine learning algorithms. Some models are relatively transparent and some are known to be uninterpretable. However, ranking the complexity of most other models which are not on either extreme sides of this spectrum, would most likely be subjective and will not represent the ground truth. Moreover, although the accuracy of these models is not our main concern, we still need to utilize it to measure the goodness of fit. Nevertheless, with all the models that we implement, we aim to achieve a reasonable accuracy without taking many extra steps to increase the accuracy of any particular model.

With the dissemination of well-known machine learning (ML) models, it is almost guaranteed that most real-world machine learning applications will take advantage of one or more of these models. Furthermore, until different fair models become widely adopted, if ever, most machine learning applications will not be built with fairness in mind unless there is an obligation to do so. This only makes sense since given the recent development in fairness in ML, it is safe to say that it is not an easy task to tackle. Therefore, we believe it is of vital importance to analyze the fairness of currently well-known ML models without any alteration. More importantly, search for any correlations between fairness and complexity of a model and find general trends which can not only help inform the AI community to implement more fairness-aware models without making fairness the first priority and implementing a fairness-first model, but also inform future research in this subject on what makes an algorithm "unfair".

\section{Background and Related Work} \label{sec:background}
A growing number of researchers has been recently exploiting fairness in AI \cite{calmon2017optimized, zafar2017fairness, kusner2017counterfactual, pleiss2017fairness, chiappa2018path, dwork2017decoupled, russell2017worlds, zhang2018fairness}. Each of these papers tend to address fairness by focusing on one or more of the components in a machine learning model's pipeline (pre-, in- and post-processing). Overall, they tend to decrease the impact of the inherited biases in the dataset by sacrificing some accuracy with respect to a certain definition of fairness. As we will explain later on, some of these models are simple, but incapable of achieving a reasonable fairness in predictions, and some other ones will generate more fair prediction, but are so complex that are almost impossible to implement.

Now, let's formalize the problem. As discussed in Section \ref{sec:intro}, discrimination occurs with respect to certain features/attributes, which we will refer to as protected features, denoted by $A$. The goal is to design a model in which membership in a particular group in a protected feature does not lead to an unfair prediction. Next, let $X$ denote other observable attributes of a given record, $U$ the set of relevant latent attributes, which are not directly observed. The goal of the machine learning algorithm will be predicting $Y$. Thus, $\hat{Y}$ is the predictor which depends on all of the attributes $A$, $X$ and $U$.

\subsection{Methods of Fairness}
In this section, we briefly cover two fairness methods just so the reader would have some idea on how fairness might be achieved in ML. However, these methods are not implemented in this study. For simplicity, assume a binary classification setting of individuals.

\textbf{\textit{Fairness through unawareness}}, \citet{grgic2016case}, suggests that for an algorithm to be fair, it may not explicitly use any of the protected feature $A$. This directly addresses disparate treatment and it was initially proposed as a baseline approach. Any mapping form $X$ to $Y$ that excludes $A$ would be considered an implementation of this method, making it extremely easy to implement in practice. However, this method does not even attempt to address disparate impact. This means any features in $X$ which are correlated with any of the protected features in $A$ will have an impact on the prediction. This would be fine if it is determined that non of the elements in $X$ contain discriminatory information similar to that of $A$, however this is rarely the case. This particular deficiency let to the proposal of the next method.

\textbf{\textit{Individual Fairness}}, \citet{dwork2012fairness}, proposes that an algorithm is fair if it gives similar prediction to similar individuals based on a \textit{given} similarity distance metric $d(., .)$. Thus if individuals $a$ and $b$ are similar, then $\hat{Y}(X^a, A^a) \approx \hat{Y}(X^b, A^b)$. Therefore, $a$ and $b$  are first each mapped to distributions over outcomes, then the distance between the two distributions over the outcomes should be less than or equal to $d(a, b)$.
This is more complicated to implement \cite{joseph2016rawlsian, louizos2015variational} and interpret as well, since the definition depends heavily on the similarity metric as well as how close the similar individuals' predictions must be. 

Recent developments in this area are mostly focusing on causal models and counterfactuals which use graphs to capture the causalities in a model \cite{kusner2017counterfactual, zhang2018fairness, chiappa2018path, russell2017worlds}. As an example, \textbf{\textit{Counterfactual Fairness}}, \citet{kusner2017counterfactual}, proposes that to be fair, $A$ should not be a cause of $\hat{Y}$ in any individual instance. In other words, changing $A$ while holding things which are not causally dependent on $A$ constant will not change the distribution of $\hat{Y}$.

\subsection{Definitions of Fairness} \label{subsec:def-of-fairness}
Here we will introduce three definitions of fairness which we will use in this study to measure the fairness of different models. Fairness in machine learning is usually a concern when the dataset contains information about individuals and thus the task is predicting $Y$ which directly or indirectly affects individuals at some level. Now, in order to better explain different definitions of fairness and for simplicity, let's assume the following prediction task. Keep in mind that non of these definitions are limited to this particular setup.

Assume a binary classification of individuals, where $Y \in {0, 1}$ and where it is beneficial for an individual to be classified as $1$. This is necessary since if being classified in one group is not preferred over the other, or classification does not in any way affect any of the individuals, there is no need to consider fairness. An example of the assumed setup can be a model which predicts whether or not a student applicant will be successful in school where the outcome is used by the admission committee to evaluate applicants. Furthermore, assume $A$ is binary feature such as gender. 

\subsubsection{\textbf{False Positive and False Negative Rates}} \label{subsec:def-of-FPR} In this context, we are interested to analyze the false positive rate (FPR) and the false negative rate (FNR) across different groups in a protected class, for example males and females in gender. This is the the most basic and a classical measure which actually tells us a lot about the prediction. Intuitively, to achieve a high accuracy we want these rates to be as small as possible and then from a fairness standpoint, we want them to be close to each other across different groups in a protected class. Having a high FPR for a particular group basically means that members of that group are receiving a preferential treatment and a high FNR means the exact opposite. This distinction is vastly important and it is why we are not interested in achieving similar prediction accuracy among different groups of a protected class.


\subsubsection{\textbf{Demographic Parity (DP)}} \label{subsec:def-of-DP} A predictor $\hat{Y}$ satisfies demographic parity if 

\begin{equation} \label{eq:dp}
P(\hat{Y} | A = 0) = P(\hat{Y} | A = 1).
\end{equation}
This means that changing the class membership of an individual in a protected feature, should not change the predicted outcome. As shown in Equation \ref{eq:dp}, this definition does not care whether or not the prediction itself was correct. 

Here is an example. If a model predicted that a male applicant will be successful (or not), the prediction should not change if the gender of the applicant is changed to female. Or on broader terms, if there are two identical applicant in the dataset with the exception of their gender, the prediction for both individuals must be the same.

An example of a model that addresses DP is proposed by \citet{zafar2017fairness}, which is done by maximizing accuracy under a fairness constraint.

\subsubsection{\textbf{Equality of Opportunity (EO)}} \label{subsec:def-of-EO} A predictor $\hat{Y}$ satisfies equality of opportunity if 

\begin{equation} \label{eq:eo}
P(\hat{Y} = 1 | A = 0, Y = 1) = P(\hat{Y} = 1 | A = 1 , Y = 1).
\end{equation}

This is the same concept as DP, however it only concerns the true positive predictions. That is why EO is also known as true positive parity. Going back to our example, if a model correctly predicts a male applicant to be successful in school, changing the gender should not change the outcome. Or again similarly, if there are two identical applicants in the dataset with the different genders, who are both known to be successful (true label is $1$), must both be classified similarly. Notice that we did not say they both must be correctly classified, since that is one way to achieve EO. If they are both incorrectly classified, then they are not even considered as Equation \ref{eq:eo} is conditioned on $Y=1$. Having this condition also limits the application of this definition to supervised learning.

\section{Dataset} \label{sec:dataset}
The initial and arguably the most important step in any fair ML setup is getting to understand the dataset. For this study we chose to empirically analyze the UCI Adult Income dataset \cite{Dua:2017} which was extracted from the 1994 Census database. This dataset has been wiedly used to evaluate newly proposed fairness methods \cite{zafar2017fairness, zafar2017parity, pleiss2017fairness, calmon2017optimized}. Every record in the dataset is an individual and the prediction task is to determine whether a person makes over \$50K a year (back in 1994), which makes it a supervised binary classification. We assume that being classified as someone who makes over \$50K a year to be advantageous for an individual. This can be a setup where the prediction is used to determine whether or not a person is eligible to receive a loan where having a higher income increases an individual's chances.

Adult dataset contains 14 features, most of which are categorical. We decided to consider \textit{race}, \textit{sex}, \textit{country of origin}, and \textit{age} to be protected features. These are all legally protected classes in U.S. \cite{calmon2017optimized}.

Furthermore, this dataset has a predefined training and testing split of $2/3$ and $1/3$, which we use in all of our analyses. There are $30162$ and $15060$ examples in the training and testing sets respectively and $3/4$ of both sets make $\leq$ \$50K a year. 

Table \ref{tab:protected-features-subset-breakdown} shows the breakdown of proportions of different subsets of each protected feature. It also should be mentioned that the subset "Other" in the "Country of Origin" includes 39 countries. Here are a few characteristics of this dataset that stand out:

\begin{itemize}
	\item Majorities are white males between the age of 18 to 40.
	\item The proportion of females who make over \$50K a year to the ones who do not, is $12.89\%$ compared to males at $45.69\%$.
	\item Same proportion for people between 18 to 29 is $6.5\%$ compared to $37.37\%$ for 30 to 39 and $60.14\%$ for 40 to 49.
	\item There are a few people in the dataset who are under 18 years of age and non of them make over \$50K a year.
	\item Over $90\%$ of the people in both training and testing sets are from the United States. 
\end{itemize}


\begin{table}[!ht]
  \centering
  \begin{tabular}{|c|c|c||c|c|}
  \cline{2-5}
  \multicolumn{1}{c|}{} & \multicolumn{2}{c||}{Training Set} & \multicolumn{2}{c|}{Testing Set} \\
  \hline 
  Subsets & $\leq$ \$50K & $>$ \$50K & $\leq$ \$50K & $>$ \$50K \\ 
  \hline
  
  
  \multicolumn{5}{|c|}{Race} \\
  \hline 
  White & 0.633 & 0.227 & 0.638 & 0.224 \\ 
  \hline 
  Black & 0.081 & 0.012 & 0.083 & 0.011 \\ 
  \hline 
  Asian-Pac-Islander  & 0.021 & 0.008 & 0.019 & 0.008 \\ 
  \hline 
  Amer-Indian-Eskimo & 0.008 & 0.001 & 0.009 & 0.001 \\ 
  \hline 
  Other & 0.007 & 0.001 & 0.007 & 0.002 \\ 
  \hline
  
  
  \multicolumn{5}{|c|}{Sex} \\
  \hline 
  Male  & 0.464 & 0.212 & 0.465 & 0.209 \\ 
  \hline 
  Female & 0.287 & 0.037 & 0.289 & 0.037 \\ 
  \hline 
  
  
  \multicolumn{5}{|c|}{Country of Origin} \\
  \hline 
  United States & 0.680 & 0.232 & 0.686 & 0.229 \\ 
  \hline 
  Mexico & 0.019 & 0.001 & 0.019 & 0.001 \\ 
  \hline 
  Other & 0.052 & 0.016 & 0.050 & 0.015 \\ 
  \hline
  
  
  \multicolumn{5}{|c|}{Age} \\
  \hline 
  Under 18 & 0.011 & 0.000 & 0.011 & 0.000 \\ 
  \hline 
  18 to 29 & 0.264 & 0.016 & 0.257 & 0.015 \\ 
  \hline 
  30 to 39 & 0.198 & 0.074 & 0.201 & 0.075 \\ 
  \hline 
  40 to 49 & 0.143 & 0.086 & 0.141 & 0.086 \\ 
  \hline 
  50 to 59 & 0.084 & 0.054 & 0.087 & 0.051 \\ 
  \hline 
  60 to 69 & 0.039 & 0.015 & 0.043 & 0.016 \\ 
  \hline 
  70 and over & 0.012 & 0.003 & 0.015 & 0.004 \\ 
  \hline 
  \end{tabular} 
  
  \caption{Protected features subset proportion breakdown of the UCI Adult Income dataset.}
   \label{tab:protected-features-subset-breakdown}
\end{table}


\section{Empirical Analyses} \label{sec:empirical-analyses}

As mentioned in Section \ref{sec:intro}, our purpose with this study is to analyze how the fairness of different well-known ML models varies as the complexity of these models increases. This section will be divided into two parts. First, we briefly explain the ML models which we implement on the dataset. Since all of these models are very well-known, we will not go into too much details. Next, we will explain how we evaluate the fairness of these models.

\subsection{Classification Models} \label{subsec:classification-models}
Overall, we fit 8 different supervised binary classification models on the UCI Adult Income dataset. As mentioned before, it is hard to correctly and objectivity rank the complexity of these models. Thus, we took a combination of two different measures. The first one is the time it took for the model to fit the training set. For this measure, all the tests are run on the same computer with the test being the only major running process. All models are fitted 100 times on the training set, each run is timed separately, then we report the average over all the runs as time to fit (TTF) in seconds. The next measure of complexity is the relative transparency of the intuition behind how the model works. We categorize them into 4 categories ($1-4$) form the most transparent to the least. Now, this tends to be more subjective and harder to empirically measure. Thus we provide our opinion on how transparent a model is. Using these two measures, we rank all models. We acknowledge that it is possible for a few of these models to be misplaced one or two rankings higher or lower than how most other people would rank them , but we believe that the overall rankings are fair and provide enough structure for us to continue our empirical analysis. 

All of these models are trained and tested on the aforementioned sets in Section \ref{sec:dataset}. Non of these models are ever trained on the testing set in any form. Here are the ML models in increasing order of complexity.

\begin{enumerate}
	\item \textit{Decision Tree (DT)} -- TTF: $0.0284 s$ - Transparency: $1$ \\
The most transparent model that we fit into our dataset is a decision tree with entropy as its criterion. The intuition behind how a decision chooses a feature for its split and how it classifies every example is well-known. Not only that, in order to make the model easier to interpret, we limited the depth of the tree to 3. It also was the second fastest model to fit the training set. As it is shown in Figure \ref{fig:desicion-tree-depth-3}, after we fit the model, non of the protected features are used to make a split in the tree.
	
\begin{figure*}
	\begin{center}
    	\centering
        \includegraphics[scale=.35]{graphics/decision-tree-depth-3}
        \caption{Decision Tree classier with entropy criterion and depth of 3. As it is shown in the tree, no decision is made based on any of the protected features.}
        \label{fig:desicion-tree-depth-3}
     \end{center}
\end{figure*}


\item \textit{Random Forest (RF)} -- TTF: $2.7403 s$ - Transparency: $1$ \\
%n_estimators=50, max_features=None, max_depth=14
In order to reasonably increase the complexity the decision tree, we decided to implement a random forest model with 50 trees, max-depth of 14 to be same as number of features, and gini as its criterion. Although it takes more time for this model to fit the data, the intuition behind how it works is still relatively easy to follow.

\item \textit{Gaussian Naive Bayes (GNB)} -- TTF: $0.0103 s$ - Transparency: $2$ \\
Here we implement a GNB model where the likelihood of features are assumed to be Gaussian, then standard deviation $\sigma$ and mean  $\mu$ of the distribution over the outcome is estimated using maximum likelihood. We also first scale all features to standard normal. This was the fastest model to fit the data, as expected, and with a relatively widely used maximizing approach the transparency is reasonable.  Thus, we placed this model third in the complexity ranking.

\item \textit{Logistic Regression (LR)} -- TTF: $0.0347 s$ - Transparency: $2$ \\
Our list of binary classification models would not have been complete without the classical LR. The following parameters seemed to yield the highest accuracy: $C=6.6$, $penalty= l1$, and $tol=0.01$ with scaling all the features to standard normal. This is a model that has been studied numerous times and the intuition on how it works is reasonably understood. Given that it took longer than GNB, it is placed 4th in the ranking.

\item \textit{K-Nearest Neighbors (KNN)} -- TTF: $0.5629 s$ - Transparency: $3$ \\
The overall intuition behind how KNN works is easy to follow at small dimensions but as the dimensions increases and we start adding weights to scaled features it becomes less and less transparent. This model uses a uniformly distributed weight on all features and looks at the nearest 15 examples. 

\item \textit{Support Vector Machine (SVM)} -- TTF: $14.8490 s$ - Transparency: $3$ \\
SVM is also one of the well-known classification models, which is not as transparent as the previous models. It also took the second longest to fit the data, thus it is ranked 3rd from the last for complexity. We first scaled all the features to a standard normal, then fitted an SVM with an RBF kernel and set $C=100$ and $gamma=0.001$, which seemed to achieve reasonable accuracy.

\item \textit{Gaussian Process Classifier (GPC)} -- TTF: $5.3hrs$ - Transparency: $4$ \\
GPC implements a Gaussian Process (GP) for probabilistic classification. It places a GP prior on a latent function which then goes through a link function, for which the integral is approximated, to obtain a probabilistic classification. This model is defiantly less transparent than any of the previously mentioned models, and it take a very very long time to fit the scaled training set. Thus, it is ranked to be the second most complex model. It should also be mentioned that this model was run only once to measure how long it took to fit the training set, however, due to the vast difference in TTF relative to the other models, it is safe to say that this is defiantly the slowest model.

\item \textit{Multilayer Perceptron (MLP)} -- TTF: $2.2169 s$ - Transparency: $4$ \\
MLP also known as neural network is notorious for being hard to interpret and often it is treated as a black box. That is why we ranked it as the least transparent model. An MLP with the following parameters is fitted to the scaled training set: activation=tan , epsilon$=0.001$, one hidden layer of size 10, solver=lbfgs, and tol=$1e^{-6}$.
\end{enumerate}


\subsection{Measuring Fairness}
In this section we will explain our approach to measure fairness. Overall, we utilized the same three definitions as described in Section \ref{subsec:def-of-fairness}. However, DP and EO simply are properties which are either satisfied or not and most of the models in Section \ref{subsec:classification-models} do not satisfy either of them. Therefore, we designed two measures in order to be able to capture how much fairness violation each model with respect to these definitions. FPR and FNR are measured as it was explained in Section \ref{subsec:def-of-FPR}.

\subsection{Demographic Parity Violation Rate (DPVR)} \label{subsec:DPVR}
As it was described in Section \ref{subsec:def-of-DP}, DP states that changing the class membership of an individual in a protected feature, should not change the predicted outcome. Therefore, in our testing set, it would be considered a violation of DP if the prediction of an individual changes, when we simply change the membership of an individual from group $A$ to $B$ in a protected feature $X$. This is precisely how we count violations with respect to DP. In a broader terms, for all subsets of a protected feature $X$, we first classify an individual with his or her given class membership. Then, we change the individual's membership to all other possible subsets and classify that person again. If the prediction changes in this process, we count that as a violation of DP for the subset that the individual originally belonged to. Next, for every subset we calculate the rate of violation to the max number of possible violation (number of examples in the subset) to get the DPVR of that subset. Next, we can average over all DPVRs in a protected feature to get the mean DPVR of the protected class itself.

\subsection{Equality of Opportunity Violation Rate (EOVR)} \label{subsec:EOVR}
This measure is intended to evaluate the violation rate with respect to EO as it is described in Section \ref{subsec:def-of-EO}. EO states that the prediction of any individual who was correctly classified as positive, should not change if we simply change his or her membership in a protected class. The process of counting violations and evaluating the rate is similar to DPVR, however instead of considering every member of a subset of a protected feature, we only consider those who were correctly classified as positive.


\section{Results} \label{sec:results}
In any ML setup, the first measure that comes to mind is accuracy. As shown in Figure \ref{fig:model-accuracy}, all models are close in accuracy. Both decision tree and random forest performed really well. RF outperformed all other models and although DT has the fifth highest accuracy, the simplicity of the model should not be ignored. Other than RF and DT, the accuracy of the models generally increase with their complexity. However, since non of the models were at any point trained on the testing set (e.g. k-fold cross validation), the accuracies are not that impressive.

%Accuracy
\begin{figure}
	\begin{center}
    	\centering
        \includegraphics[scale=.65]{graphics/models-accuracy}
        \caption{Prediction accuracy of all models in Section \ref{subsec:classification-models}.}
        \label{fig:model-accuracy}
     \end{center}
\end{figure}

Next, we take a look at mean FPR across the models. As it is shown in Figure \ref{fig:model-fpr}, overall we can see an increase in mean FPR in the protected features as the complexity of the models increase. Nevertheless, it is not a strong correlation. There are exceptions to this. For instance, DT, our least complex model, has the second worst FPR for country of origin.

%FPR
\begin{figure}
	\begin{center}
    	\centering
        \includegraphics[scale=.65]{graphics/models-fpr}
        \caption{Mean false positive rate of protected features of all models in Section \ref{subsec:classification-models}}
        \label{fig:model-fpr}
     \end{center}
\end{figure}

Moving on to the mean FNR, as it is shown in Figure \ref{fig:model-fnr}, it seems that the models with moderate complexity actually performed worse than the most and least complex models. Nonetheless, RF had the most fair predictions with respect to FNR. Surprisingly, GNB had the worst mean FNR among the models in all protected classes. On the other hand, our top two complex models, GPC and MLP had an average FPR for race, sex and age.

%FNR
\begin{figure}
	\begin{center}
    	\centering
        \includegraphics[scale=.65]{graphics/models-fnr}
        \caption{Mean false negative rate of protected features of all models in Section \ref{subsec:classification-models}}
        \label{fig:model-fnr}
     \end{center}
\end{figure}

With respect to our other two fairness measures, DPVR in Figure \ref{fig:model-dpvr}, and EOVR in Figure \ref{fig:model-eovr}, we can clearly pick our most fair model, which is the decision tree. The fact that DT did not use any of the protected features in its tree, automatically satisfies both demographic parity and equality of opportunity. This means a violation rate of zero for all the protected features. However, this is a bit concerning given DT's FPR and FNR. Mean false positive rate of DT with respect to the country of origin was the second worst, and its FNR was on the higher end of the spectrum. This is a clear sign that satisfying DP and EO does not necessary cause fairness with respect to other measures. Not only that, it also speaks to the complexity of fairness and how hard it is to achieve.

Furthermore, KNN has the worst DPVR and EOVR among all the models with respect to race, sex and country. One way of interpreting this would be the fact that changing the membership of an individual in a protected class, will most likely change the K nearest neighbors around that individual, causing the prediction to change. There are some similarities between how KNN classifies individuals and stereotyping people. This is exactly what fairness in AI is trying to prevent.

%DPVR
\begin{figure}
	\begin{center}
    	\centering
        \includegraphics[scale=.65]{graphics/models-dpvr}
        \caption{Mean demographic parity violation rate of protected features of all models in Section \ref{subsec:classification-models}}
        \label{fig:model-dpvr}
     \end{center}
\end{figure}

Another interesting observation here is unacceptably high mean EOVR for age as it is shown in Figure \ref{fig:model-eovr}.  Except DT which is always zero and GNB which is close to $20\%$, the violation rate for all other models exceed $40\%$ and in the case of GPC it even exceeds $60\%$. One way to explain why this is happening can be the really low rate of positive examples among the people between 18 to 29 as it was explained in Section \ref{sec:dataset}. Since $93.4\%$ of individuals in this group in the training set make less than or equal to \$50K a year, when we change the age of people in other groups to be in this range, the prediction changes. To confirm this, instead of changing the age of individuals to all other different subsets, we only changed them to be between 18 to 29 and although the mean EOVR was not as high as Figure \ref{fig:model-eovr}, it was really close, suggesting this to be the cause.

%EOVR
\begin{figure}
	\begin{center}
    	\centering
        \includegraphics[scale=.65]{graphics/models-eovr}
        \caption{Mean equality of opportunity violation rate of protected features of all models in Section \ref{subsec:classification-models}}
        \label{fig:model-eovr}
     \end{center}
\end{figure}

Now, going back to the trade-off between fairness and complexity in both DPVR and EOVR, if we exclude KNN, there exists an upward trend in violations and complexity. However, it is not nearly strong enough for us to confidently draw conclusions. There a too many exceptions to this trade-off which makes it hard to interpret. There are instances of simple models which are fair, and complex models that are more fair than some of the less complex ones (e.g. SVM in Figure \ref{fig:model-dpvr}).

Overall, given all different measures of fairness, it is reasonable to say that there does exists a weak trade-off between fairness and complexity. However, a more detailed analysis on the subset level is needed to find stronger patterns.

\input{Conclusion}

\bibliographystyle{ACM-Reference-Format}
\bibliography{bibliography}

\end{document}
